
% ======================
% Einbinden von Packages
% ======================

% Layout
\usepackage{a4}									% Ausgabeformat
\usepackage{lastpage}						% Referenz auf letzte Seite des Dokuments verfügbar
\usepackage{lscape}							% Querformat einer Seite durch den Aufruf \begin{landscape}
\usepackage{rotating}						% Drehen von Gleitobjekten 

% Spracheinstellungen
\usepackage[english]{babel}
   	% neue deutsche Rechtschreibung, deutsche Spracheinstellungen
\usepackage[ansinew]{inputenc}  % direkte Eingabe von Umlauten über Tastatur
\usepackage[T1]{fontenc}				% direkte Eingabe von Umlauten über Tastatur
%\usepackage[utf8]{inputenc}
\usepackage{lmodern}


% Schriften
\usepackage{helvet}         		% Helvetica als serifenlose Schriftart verwenden
\usepackage[bf]{caption}    		% Formatierung von Tabellen- und Bildbeschriftungen
\usepackage{verbatim}						% Ausgabe des Textes wie im Editor dargestellt, Tex-Direktiven werden nicht berücksichtigt
%\setkomafont{sectioning}{\normalfont\bfseries} % Kapitelüberschriften mit Serifen
\setkomafont{sectioning}{\sffamily\bfseries} % Kapitelüberschriften ohne Serifen
\setkomafont{captionlabel}{\normalfont\bfseries}
\setkomafont{descriptionlabel}{\normalfont\bfseries}


% Grafiken
\usepackage{graphics}						% Standardgrafikpaket
\usepackage{graphicx} 					% Erweiterete Grafikoptionen
\usepackage{subfigure}					% Unterabbildungen zulassen
\usepackage{epsfig}							% Einbinden von Grafiken
\usepackage{psfrag}    					% Einfügen von Latex-Schriften in eps-Bilder
\usepackage{svg}
% \usepackage{tikz}								% Package zum Zeichnen von Grafiken
% \usepackage{pgfplots}						% Package zum Zeichnen von Plots, greift auf tikz zu
\usepackage{epstopdf}		% wird nur mit PDFLatex benötigt, damit auch eps-Bilder verwendet werden können

% Paket zum Erweitern der Tabelleneigenschaften
\usepackage{array}							% erweiterte Einstellmöglichkeiten für Spaltenformate
\usepackage{dcolumn}						% Ausrichtung des Dezimalpunktes
\usepackage{multirow}						% mehrere Zeilen zusammefassen
\usepackage{tabularx}   				% Erweiterte Tabellen Optionen
\usepackage{booktabs}						% normgerechte Tabellen
\usepackage{longtable}					% Mehrseitige Tabellen

% Formeln
\usepackage{amsmath}    				% Mathematik-Modus
\usepackage{amsfonts}						% Schriftarten für Mathematik-Modus
% \usepackage{dsfont}							% Zahlenmengen-Schriftart
\usepackage{amssymb}    				% Symbole für Mathematik-Modus
\usepackage{wasysym}						% >>,<<,approx
\usepackage{upgreek}						% nichtkursive griechische Buchstaben
% \usepackage{trfsigns}						% Transformationssymbole

% Aufzählung
\usepackage{mdwlist} 						% Aufzählung unterbrechen mit suspend{enumerate} und resume{enumerate}
\usepackage{paralist}						% Verwendung von unterschiedichen Zeichen in einer Aufzählung


% Sonstiges
\usepackage{ifthen}							% Abfragen und bedingte Formatierung
\usepackage[
  hidelinks,        % remove colored boxes around links
  linktocpage,      % make the page number, not the text, the clickable target
  breaklinks=true,  % allow links to break across lines
]{hyperref}

\usepackage{xspace}    					% für Leerzeichen in Mathe
 
\usepackage{bibgerm}						% Literaturverzeichnis

\usepackage{tikz}		% fuer circuitikz Schaltpläne
\usepackage[european, betterproportions, americaninductors]{circuitikz}

\usepackage{pgfplots}   % um Plots darzustellen          
\pgfplotsset{/pgf/number format/.cd, use comma}  % Komma bei Achsenbeschriftung
\usepgfplotslibrary{groupplots}

% Einheiten
\usepackage{siunitx}  % bei newCommands_user Zeile 21 renewcommand statt newcommand, Zeile 85 auskommentiert
\sisetup{locale = DE} % für Komma statt Punkt
\usepackage{iftex}
\ifPDFTeX
  % pdflatex is in use:
  % Do not load fontspec.
  % Instead, rely on packages like inputenc, fontenc, and lmodern (which you already have)
\else
  % XeLaTeX or LuaLaTeX is in use:
  \usepackage{fontspec}
  % Optionally, set your main font here:
  % \setmainfont{Times New Roman}
\fi
% Core Packages
\usepackage{tikz}          % For TikZ graphics
\usetikzlibrary{positioning, arrows.meta, shapes.geometric, fit}
\usepackage{graphicx}      % For figure handling
\usepackage{xcolor}        % For color shading (e.g., gray!30)
\usepackage[htt]{hyphenat}
% TikZ Libraries
\usetikzlibrary{arrows.meta}
\usepackage{listings}
\lstdefinestyle{pythonstyle}{
    language=Python,
    basicstyle=\footnotesize\ttfamily,
    keywordstyle=\color{blue},
    commentstyle=\color{gray},
    stringstyle=\color{red},
    numbers=left,
    stepnumber=1,
    numbersep=5pt,
    breaklines=true,
    frame=single,
    tabsize=4,
    showstringspaces=false,
}
