
% ======================
% Einbinden von Packages
% ======================

% Layout
\usepackage{a4}									% Ausgabeformat
\usepackage{lastpage}						% Referenz auf letzte Seite des Dokuments verf�gbar
\usepackage{lscape}							% Querformat einer Seite durch den Aufruf \begin{landscape}
\usepackage{rotating}						% Drehen von Gleitobjekten 

% Spracheinstellungen
\usepackage[ngerman]{babel}    	% neue deutsche Rechtschreibung, deutsche Spracheinstellungen
\usepackage[ansinew]{inputenc}  % direkte Eingabe von Umlauten �ber Tastatur
\usepackage[T1]{fontenc}				% direkte Eingabe von Umlauten �ber Tastatur


% Schriften
\usepackage{helvet}         		% Helvetica als serifenlose Schriftart verwenden
\usepackage[bf]{caption}    		% Formatierung von Tabellen- und Bildbeschriftungen
\usepackage{verbatim}						% Ausgabe des Textes wie im Editor dargestellt, Tex-Direktiven werden nicht ber�cksichtigt
%\setkomafont{sectioning}{\normalfont\bfseries} % Kapitel�berschriften mit Serifen
\setkomafont{sectioning}{\sffamily\bfseries} % Kapitel�berschriften ohne Serifen
\setkomafont{captionlabel}{\normalfont\bfseries}
\setkomafont{descriptionlabel}{\normalfont\bfseries}


% Grafiken
\usepackage{graphics}						% Standardgrafikpaket
\usepackage{graphicx} 					% Erweiterete Grafikoptionen
\usepackage{subfigure}					% Unterabbildungen zulassen
\usepackage{epsfig}							% Einbinden von Grafiken
\usepackage{psfrag}    					% Einf�gen von Latex-Schriften in eps-Bilder
% \usepackage{tikz}								% Package zum Zeichnen von Grafiken
% \usepackage{pgfplots}						% Package zum Zeichnen von Plots, greift auf tikz zu
\usepackage{epstopdf}		% wird nur mit PDFLatex ben�tigt, damit auch eps-Bilder verwendet werden k�nnen

% Paket zum Erweitern der Tabelleneigenschaften
\usepackage{array}							% erweiterte Einstellm�glichkeiten f�r Spaltenformate
\usepackage{dcolumn}						% Ausrichtung des Dezimalpunktes
\usepackage{multirow}						% mehrere Zeilen zusammefassen
\usepackage{tabularx}   				% Erweiterte Tabellen Optionen
\usepackage{booktabs}						% normgerechte Tabellen
\usepackage{longtable}					% Mehrseitige Tabellen

% Formeln
\usepackage{amsmath}    				% Mathematik-Modus
\usepackage{amsfonts}						% Schriftarten f�r Mathematik-Modus
% \usepackage{dsfont}							% Zahlenmengen-Schriftart
\usepackage{amssymb}    				% Symbole f�r Mathematik-Modus
\usepackage{wasysym}						% >>,<<,approx
\usepackage{upgreek}						% nichtkursive griechische Buchstaben
% \usepackage{trfsigns}						% Transformationssymbole

% Aufz�hlung
\usepackage{mdwlist} 						% Aufz�hlung unterbrechen mit suspend{enumerate} und resume{enumerate}
\usepackage{paralist}						% Verwendung von unterschiedichen Zeichen in einer Aufz�hlung


% Sonstiges
\usepackage{ifthen}							% Abfragen und bedingte Formatierung


\usepackage{xspace}    					% f�r Leerzeichen in Mathe
 
\usepackage{bibgerm}						% Literaturverzeichnis
