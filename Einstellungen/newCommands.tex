% Floatparameter f�r ein g�nstigeres Platzieren von floatobjekten
\renewcommand{\floatpagefraction}{.75} %default .5
\renewcommand{\textfraction}{.15}
\renewcommand{\topfraction}{.8}
\renewcommand{\bottomfraction}{.5}

% Verschiedene Verweisarten
% Erweiterung zweisprachig, siehe TdS Aufgabensammlung
\iflanguage{ngerman}{
\newcommand{\Equ}[1]{Gleichung~(\ref{equ:#1})}
\newcommand{\Abb}[1]{Abbildung~\ref{fig:#1}}
\newcommand{\abb}[1]{Abb.~\ref{fig:#1}}
\newcommand{\ABB}[2]{Abbildungen~\ref{fig:#1} und~\ref{fig:#2}}
\newcommand{\Abbildungen}[2]{Abbildungen~\ref{fig:#1} bis~\ref{fig:#2}}
\newcommand{\Tab}[1]{Tabelle~\ref{tab:#1}}
\newcommand{\Kap}[1]{Kapitel~\ref{chap:#1}}
\newcommand{\Absch}[1]{Abschnitt~\ref{sec:#1}}
\newcommand{\UAbsch}[1]{Abschnitt~\ref{subsec:#1}}
\newcommand{\App}[1]{Anhang~\ref{app:#1}}
\newcommand{\AppAbsch}[1]{Anhang~\ref{sec:#1}}
% Deutsche Bezeichnungen im Literaturverzeichnis
\newcommand{\btxeditionlong}[1]{Auflage}
}
{
\newcommand{\Equ}[1]{formula~(\ref{equ:#1})}
\newcommand{\Abb}[1]{figure~\ref{fig:#1}}
\newcommand{\abb}[1]{fig.~\ref{fig:#1}}
\newcommand{\ABB}[2]{figure~\ref{fig:#1} and~\ref{fig:#2}}
\newcommand{\Abbildungen}[2]{figures~\ref{fig:#1} to~\ref{fig:#2}}
\newcommand{\Tab}[1]{table~\ref{tab:#1}}
\newcommand{\Kap}[1]{chapter~\ref{chap:#1}}
\newcommand{\Absch}[1]{section~\ref{sec:#1}}
\newcommand{\UAbsch}[1]{section~\ref{subsec:#1}}
\newcommand{\App}[1]{appendix~\ref{app:#1}}
\newcommand{\AppAbsch}[1]{appendix~\ref{sec:#1}}
}

% Tabellenspalten mit Flattersatz setzen, muss \\ vor (z.B.) \raggedright geschuetzt werden:
\newcommand{\PBS}[1]{\let\temp=\\#1\let\\=\temp} 

