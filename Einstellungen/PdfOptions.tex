
% ---------------------------------------------------------------------------
% Pdf Optionen
% ---------------------------------------------------------------------------
%
% Type 1 Fonts f�r bessere darstellung in PDF verwenden.
%
%\usepackage{mathptmx}           % Times + passende Mathefonts
%\usepackage[scaled=.92]{helvet} % skalierte Helvetica als \sfdefault
%\usepackage{courier}            % Courier als \ttdefault


%
% Package f�r Farben im PDF
%
\usepackage{color}
\definecolor{hellgrau}{rgb}{0.9,0.9,0.9}

% Paket f�r Links innerhalb des PDF Dokuments

\definecolor{LinkColor}{rgb}{0,0,0.5}

\definecolor{LinkColor}{rgb}{0,0,0}

% Kapitel, Abbildungen, Tabellen und Literatur verlinken
\usepackage[
	linktocpage,					% Setzt Links in Verzeichnissen auf Seitenzahl --> keine Probleme mit Umbr�chen
	ps2pdf,
	colorlinks=false,			% Links farbig markieren; Druckversion: false, PDF-Version: true
	linkcolor=blue,				% Linkfarbe
	pdftitle={\worktitle},
	pdfauthor={\student},
	pdfsubject={\worksubject, \student},
	pdfkeywords={\keywords}]
{hyperref}


% Anmerkung: Listings sind nicht kompatibel mit 
% Paket um Listings sauber zu formatieren.

\usepackage[savemem]{listings}
\lstloadlanguages{TeX}
\lstloadlanguages{Matlab}


% Listing Definitionen f�r PHP Code

\definecolor{lbcolor}{rgb}{0.85,0.85,0.85}
\lstset{language=[LaTeX]TeX,
	numbers=left,
	stepnumber=1,
	numbersep=5pt,
	numberstyle=\tiny,
	breaklines=true,
	breakautoindent=true,
	postbreak=\space,
	tabsize=2,
	basicstyle=\ttfamily\footnotesize,
	showspaces=false,
	showstringspaces=false,
	extendedchars=true,
	backgroundcolor=\color{lbcolor}}

